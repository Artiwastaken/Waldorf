% Egyenletrendszerek – jegyzet
\documentclass[12pt,a4paper]{article}

% Nyelv és kódolás
\usepackage[T1]{fontenc}
\usepackage{lmodern}
\usepackage[utf8]{inputenc}
\usepackage[magyar]{babel}

% Matematika
\usepackage{amsmath,amssymb}
\usepackage{enumitem}

% Oldalbeállítások (egyszerű)
\usepackage[margin=2.5cm]{geometry}

\title{Egyenletrendszerek}
\author{10.\ osztály}
\date{\today}

\begin{document}
\maketitle

\section*{Bevezetés}
\noindent Ebben a anyagban röviden áttekintjük a egyenletrendszerek megoldásának módszereit:
 behelyettesítés, összeadás-kivonás (elimináció), valamint grafikus módszerek alapjai.
\\\\
\noindent A következő a feladat:
Béla 3 évvel ezelőtt háromszor annyi idős volt, mint András, három év múlva pedig kétszer annyi idős lesz. Hány évesek most?
\\\\
\noindent A két ismeretlenes egyenletrendszer felírásához jelöljük Béla életkorát $y$-nal, Andrásét pedig $x$-szel. 
Ekkor a feladat szövegéből az alábbi egyenleteket írhatjuk fel:
\begin{align*}
	(1)\ y-3 &= 3(x-3)\\
	(2)\ y+3 &= 2(x+3)
\end{align*}
\noindent A zárójeleket kibontjuk és átalakítjuk:
\begin{align*}
	y - 3 &= 3x - 9 \quad \Rightarrow \quad y=3x-6\\ 
	y + 3 &= 2x + 6 \quad \Rightarrow \quad y=2x+3
\end{align*}

\noindent \textbf{Grafikus} megoldáshoz oldjuk meg az egyenleteket pontonként és ábrázoljuk a kapott pontokat koordináta-rendszerben. 
Ahol a pontokat összekötő két egyenes metszi egymást, ott van a megoldás.
\\\\
\textbf{Behelyettesítéses} megoldás esetén azt csináljuk, hogy az egyik egyenletből kifejezzük az egyik ismeretlent, és ezt behelyettesítjük a másik egyenletbe. 
Például az első egyenletből kifejezzük $y$-t:
\begin{align*}
	y &= 3x - 6 
\end{align*}

\noindent Behelyettesítjük a második egyenletbe az $x$ helyére:
\begin{align*}	
	3x - 6 &= 2x + 3 \\
	3x - 2x &= 3 + 6 \\
	x &= 9
\end{align*}

\noindent Ezt visszahelyettesítjük az első egyenletbe:
\begin{align*}
	y &= 3 \cdot 9 - 6 = 27 - 6 = 21
\end{align*}

\noindent Tehát András 9 éves, Béla pedig 21 éves.
\\\\
\newpage
\noindent További feladatok:
\begin{enumerate}[label=\alph*]
  \item (1) $y=2x+1$\\
		(2) $y=-x+4$
  \item (1) $2x+3y=4$\\
		(2) $5x-6y=-7$
  \item (1) $2x+y=5$\\
		(2) $-x+y=-1$
\end{enumerate}

\end{document}

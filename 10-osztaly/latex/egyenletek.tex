% Egyenletek – jegyzet
\documentclass[12pt,a4paper]{article}

% Nyelv és kódolás
\usepackage[T1]{fontenc}
\usepackage{lmodern}
\usepackage[utf8]{inputenc}
\usepackage[magyar]{babel}

% Matematika
\usepackage{amsmath,amssymb}
\usepackage{enumitem} % betűs sorszámozás

% Oldalbeállítások (egyszerű)
\usepackage[margin=2.5cm]{geometry}

\title{Egyenletek}
\author{10.\ osztály}
\date{\today}

\begin{document}
\maketitle

\section*{Bevezetés}
\noindent Ebben az anyagban áttekintjük a lineáris egyenletek megoldásának módszereit, 
úgy mint az egyenletek átalakítása, a mérlegelv alkalmazása, és a grafikus megoldást.
 \\\\
\noindent Nézzük a következő feladatot: Melyik az a szám, amelynek ha a 10-szereséhez 9-et hozzáadunk, 
akkor ugyanazt a számot kapjuk, mintha a hatszorosához 21-et hozzáadnánk?
\\\\
\noindent A szöveges feladat egyenletként felírva:
\begin{align*}
    10x + 9 &= 6x +21
\end{align*}
\noindent \textbf{Mérlegelvvel} való megoldásához az egyenletet úgy kell átalakítani, hogy az ismeretlen $x$ egyedül álljon az egyik oldalon,
a másik oldalon pedig csak számok legyenek. 
\\\\
\noindent \textbf{Grafikus} megoldáshoz oldjuk meg az egyenlet két oldalát pontonként és ábrázoljuk a kapott pontokat koordináta-rendszerben.
\\\\
\noindent \textbf{Szöveges} feladatoknál elsőnek az ismeretlent $(x)$ keressük meg, majd ez alapján felírjuk az egyenletet.
\\\\
\noindent További feladatok:
\noindent
\begin{enumerate}[label=\alph*]
  \item \quad $5x + 4 = 3x + 18$
  \item \quad $2(x-3) = 14 - x+1$
  \item \quad $\dfrac{x}{3} + \dfrac{x}{2} = 10$
\end{enumerate}

\end{document}
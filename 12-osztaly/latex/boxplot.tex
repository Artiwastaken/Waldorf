% Box plot – jegyzet
\documentclass[12pt,a4paper]{article}

% Nyelv és kódolás
\usepackage[T1]{fontenc}
\usepackage{lmodern}
\usepackage[utf8]{inputenc}
\usepackage[magyar]{babel}

% Matematika és felsorolás
\usepackage{amsmath,amssymb}
\usepackage{enumitem}

% Oldalbeállítások (egyszerű)
\usepackage[margin=2.5cm]{geometry}

\title{Box plot (doboz-ábra)}
\author{12.\ osztály}
\date{\today}

\begin{document}
\maketitle

\subsubsection*{Bevezetés}
\noindent A box plot (doboz-ábra) egy adat\-eloszlást szemléltető grafikon, általában a számegyenesen ábrázoljuk és a mintából/halmazból fontos pontot ábrázol:
\emph{minimum}, \emph{alsó kvartilis} ($Q_1$), \emph{medián} ($Q_2$), \emph{felső kvartilis} ($Q_3$) és \emph{maximum}. 
A doboz a $Q_1$ és $Q_3$ közötti tartományt jelöli, benne egy vonal a mediánt jelöli. 
\\

\subsubsection*{A fontos elemek}
\noindent Első lépésként sorba kell rendezni a 
\begin{itemize}
  \item $Minimum:$ a legkisebb érték a mintában.
  \item $Medián$ ($Q_2$): a középső érték, ha páratlan számú elem van; 
  páros számú elem esetén a két középső érték átlaga.
  \item $Alsó\ kvartilis$ ($Q_1$): a medián alatti értékek mediánja.
  \item $Felső\ kvartilis$ ($Q_3$): a medián feletti értékek mediánja.
  \item $Maximum:$ a legnagyobb érték a mintában.
  \item $IQR$ (interkvartilis terjedelem): $IQR = Q_3 - Q_1$, másnéven a doboz.
\end{itemize}

\subsubsection*{Gyakorló feladatok}
\noindent Készítsd el az alábbi minták box plotját. 
Számold ki az mediánt, alsó és felső kvartilist, minimumot és maximumot.
\\
\begin{enumerate}[label=\alph*]
  \item \noindent Minta: $\{\,1,\ 1,\ 2,\ 2,\ 2,\ 2,\ 3,\ 4,\ 4,\ 5,\ 5,\ 6\,\}$
  \item \noindent Minta: $\{\,12,\ 12,\ 13,\ 14,\ 14,\ 15,\ 16,\ 17,\ 17,\ 18,\ 20,\ 22,\ 25\,\}$
  \item \noindent Minta: $\{\,7,\ 4,\ 12,\ 8,\ 7,\ 3,\ 9,\ 4,\ 15,\ 8,\ 5,\ 6,\ 7\,\}$
  \item \noindent Minta: $\{\,24,\ 33,\ 23,\ 29,\ 25,\ 21,\ 27,\ 23,\ 24,\ 26,\ 29,\ 25,\ 22,\ 35,\ 23\,\}$
  \item \noindent Minta: $\{\,16,\ 13,\ 24,\ 11,\ 15,\ 30,\ 20,\ 13,\ 17,\ 10,\ 18,\ 14,\ 15,\ 13,\ 16\,\}$
\end{enumerate}

\end{document}
